\section{Experience}
	
	\cvitem{March 2010 -- Present}{4PSA}{SQA Engineer}{}{}{}


	\subsection{SQA VoIP engineer}
		\begin{itemize}
	
			\item expertise area: voip 	
	
			\item responsabilities: 
				\begin{itemize}
					\item testing of the various components of the voip application (unified communications server) 
					\item testing scenarios 
					\item verifing, finding, checking bugs, or find solutions for fixing them
					\item developing and/or improving testing tools
					\item admin of the CI system using atlassian bamboo
				\end{itemize}
			\item working experience with:
				\begin{itemize}				
					\item Programming languages: C, C++, php, lua, erlang, bash
					\item Products: Asterisk, Kamailio, Redis, Ejabberd, Rabbitmq
				\end{itemize}

		\end{itemize}

	\subsection{Cisco instructor}
		\begin{itemize}
			\item period: May 2005 – July 2007 
			\item institutions: 
			\begin{itemize}
			 	\item CISCO Academy – UPB from "The Faculty of Automatics and Computer Science, Polytechnical University of Bucharest", 
				\item "Advanced Information Technology Consortium" Academy from Faculty of Automatics and Computer Science – Polytechnical University of Bucharest,
			\end{itemize}
			\item attributions:
			\begin{itemize}
			 	\item elaborating of prezentations with the purpose of being held during clases,
				\item holding presentations in front of classes of 15-20 students,
				\item guiding the students and supervising them during laboratory hours,
				\item support for the computer networks infastructure needed at different events organized by The Faculty of Automatics and Computer Science, Polytechnical University of Bucharest,
				\item evaluating the students,
				\item conducting interviews with the purpuse of selecting the aspirant students for the CISCO Networking Academy courses (CCNA and Network Security) – Cisco Networking Academy – Faculty of Automatics and Computer Science – Polytechnical University Bucharest. 
	
			\end{itemize}
		\end{itemize}
	\subsection{Network and system engineer}
		\begin{itemize}
 			\item position: member of the technical stuff that assured technical support for the ACM Programming Contest (Network Solution Provider)
			\begin{itemize}
		 		
				\item period and events: 
				\begin{itemize}
					\item October 2005 ACM Southeastern Europe Programming Contest, Bucharest, România
					\item October 2006 ACM Southeastern Europe Programming Contest, Bucharest, România
				\end{itemize}
				\item attributions: designing and implementing the network needed for the good run of the contest, 
				\item configuring for maximum security the switches and routers used so that the contest is held in a fair way,

			\end{itemize}
			\item position: technical support 
				\begin{itemize}
				 \item event: CISCO Expo 2006, Bucharest, România
                                 \item as a member of the technical crue from the Cisco Academy from the Faculty of Automatics and Computer Science, PUB.
                                \end{itemize}
			\item position: member of the netgroup at the Villum Kann Rasmussen Kollegiet (danish dormitory)
				\begin{itemize}
					\item period: March 2008 - October 2009 
					\item attributions: responsable for managing, monitoring, troubleshooting the network equipment, the local server, installing new equipment, providing Internet connection to the inhabitants of the kollegium, assuring that the access policies are respected.
				\end{itemize}
			\item part-time student job \textbf{Motorola Denmark} in the network transport team
				\begin{itemize}
					\item period: May 2008 - October 2009
					\item attributions: 
					\begin{itemize}
						\item managing the development lab of the network transport team, 
						\item performance testing for TETRA transport network, 
						\item development of a packet generating testing system, 
						\item development of technical guidelines for new features
					\end{itemize}
				\end{itemize}
		\end{itemize}	
	

	\subsection{Independent Projects}
		\subsubsection{Web programming}
		\begin{itemize}
			\item \textbf{parohiaortodoxacopenhaga.dk} - Web design and programming for the web site of the Romanian Orthodox Church in Copenhagen 
		\end{itemize}

		\subsubsection{Graphical design}
		\begin{itemize}
			\item Flyers for \textbf{Seishinkan Dojo} - Graphical design of the flyers for the martial arts club Seishinkan in Copenhagen
		\end{itemize}
		
		
		
\section{Events and conferences attended}
	\begin{itemize}
		\item \textbf{CISCO Expo:}  24-25 March 2004 – guest,
		\item \textbf{IDC IT Security Roadshow 2006:} 09.02.2006, representing CISCO Romania,
		\item \textbf{CISCO Expo:} 10-11 April 2006 – Bucharest – Romania – guest and tech support team of the event,
		\item \textbf{IDC Storage Roadshow CEE 2006}: 09.05.2006 – representing CISCO Romania. 
		\item \textbf{IDC IT Security Roadshow 2007} 
		\item \textbf{CISCO Expo 2007} – on an invitation by Cisco Romania
		\item \textbf{Cisco RoNetAcad Conference 2007} – representing CATC Romania
		\item \textbf{Mobile Linux Hacker Workshop} - Nokia Denmark - October 2007
		\item \textbf{Summerschool in Symbolic dynamics and homeomorphisms of the Cantor set} - University of Copenhagen, June 23-27, 2008  
	\end{itemize}
