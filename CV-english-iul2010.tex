\documentclass[a4paper,12pt,openany]{article}
\usepackage{verbatim,afterpage}
\usepackage[pdftex,colorlinks]{hyperref}
\usepackage{ucs}
\usepackage[utf8x]{inputenc}
\usepackage{graphicx}

\renewcommand{\labelitemi}{$\bullet$}
\renewcommand{\labelitemii}{$\bullet$}
\renewcommand{\labelitemiii}{$\bullet$}



\title{\begin{huge}Curriculum Vitae\end{huge}}
\author{\begin{large}Costin-Tiberiu Radu\end{large}}


\begin{document}
\maketitle
\section{Personal information}
	\begin{itemize}
 		\item Family name:	RADU
		\item Given name:	Costin-Tiberiu
		\item Date of birth:	Octomber, 2nd, 1984,
		\item Address:		Laborator 131, Bl S2, Ap 21, Sector 3, București, România, Europe
		\item Phone number:	0744723580
		\item E-mail:		costin.tiberiu.radu@gmail.com
		\item Nationality:	romanian. 
	\end{itemize}

\section{Studies}
	\begin{itemize}
	 	\item 1991-1999: Şcoala Generală Nr 88, Bucureşti,
		\item 1999-2003: National College „Cantemir-Vodă”, mathematics-informatics profile
		\item 2003-2007: Mathematics and Informatics Faculty, University of Bucharest, Informatics specialisation track
		\item 2007-2009: Technical University of Denmark, Master of Science in Telecommunications \\
			- finished with a thesis about a gateway between SIP and XMPP written in Erlang
		\item 2009-2010: SNSB - Școala Normală Superioară București
	\end{itemize}

\section{Extra studies}
	\subsection{Italian language course}
		\begin{itemize}
		 	\item period: january-may 1994
			\item place: Scuola italiana "Aldo Moro" - Bucureşti
		\end{itemize}
	\subsection{CISCO Networking Academy courses}
		\begin{itemize}
		 	\item CCNA  (CISCO Certified Network Associate) - modules 1 - 4. 
			\begin{itemize}
				\item period: December 2000 - March 2002,
				\item place: “Cantemir Voda - National College”  Academy – within Cantemir-Vodă College Bucureşti.                                     
			\end{itemize}
			\item Network Security
			\begin{itemize}
				\item period: November 2005 – May 2006,
				\item place: “Advanced Information Technology Consortium” Academy
			\end{itemize}
			\item CCNA (CISCO Certified Network Associate), instructor classes, modules 1 - 4:
			\begin{itemize}
 				\item period April 2005 - September 2005,
				\item place: “Advanced Information Technology Consortium” Academy from Faculty of Automatics and Computer Science – Polytechnical University of Bucharest,
			\end{itemize}
			\item Network Security - instructor classes
			\begin{itemize}
			 	\item period: May 2006 – August 2006,
				\item place: “Advanced Information Technology Consortium” Academy from Faculty of Automatics and Computer Science – Polytechnical University of Bucharest,
			\end{itemize}
		\end{itemize}
\newpage
\section{Certifications}
		\begin{itemize}
		 	\item Professional certificate "Competenţe de operare pe calculator" ("Competences on operating the computer") – date: May 2003,
			\item “Certificate in Advanced English”, University of Cambridge ESOL – date: 13.02.2004,
			\item “International English Language Testing System” - IELTS - academic test – date: 12.01.2007,
			\item CISCO Certificates as a result of the courses listed above.
		\end{itemize}
	
\section{Foreign languages}
		\begin{itemize}
		 	\item English – spoken: \textit{\textbf{advanced}}, reading: \textbf{\textit{advanced}}, written: \textbf{\textit{advanced}} – attested by “Certificate in Advanced English”, University of Cambridge ESOL, and “International English Language Testing System” - academic test,
			\item French – spoken: \textbf{\textit{medium}},    reading: \textbf{\textit{medium}}, written: \textbf{\textit{medium}},
			\item Italian – spoken: \textbf{\textit{medium}}, reading: \textbf{\textit{medium}}, written: \textbf{\textit{beginner}}.
			\item Danish - spoken: \textbf{\textit{beginner}}, reading: \textbf{\textit{beginner}}, written: \textbf{\textit{beginner}} 
		\end{itemize}
\newpage
\section{Experience}
	\subsection{Cisco instructor}
		\begin{itemize}
			\item period: May 2005 – July 2007 
			\item institutions: 
			\begin{itemize}
			 	\item CISCO Academy – UPB from "The Faculty of Automatics and Computer Science, Polytechnical University of Bucharest", 
				\item "Advanced Information Technology Consortium" Academy from Faculty of Automatics and Computer Science – Polytechnical University of Bucharest,
			\end{itemize}
			\item attributions:
			\begin{itemize}
			 	\item elaborating of prezentations with the purpose of being held during clases,
				\item holding presentations in front of classes of 15-20 students,
				\item guiding the students and supervising them during laboratory hours,
				\item support for the computer networks infastructure needed at different events organized by The Faculty of Automatics and Computer Science, Polytechnical University of Bucharest,
				\item evaluating the students,
				\item conducting interviews with the purpuse of selecting the aspirant students for the CISCO Networking Academy courses (CCNA and Network Security) – Cisco Networking Academy – Faculty of Automatics and Computer Science – Polytechnical University Bucharest. 
	
			\end{itemize}
		\end{itemize}
	\subsection{Network and system engineer}
		\begin{itemize}
 			\item position: member of the technical stuff that assured technical support for the ACM Programming Contest (Network Solution Provider)
			\begin{itemize}
		 		
				\item period and events: 
				\begin{itemize}
					\item October 2005 ACM Southeastern Europe Programming Contest, Bucharest, România
					\item October 2006 ACM Southeastern Europe Programming Contest, Bucharest, România
				\end{itemize}
				\item attributions: designing and implementing the network needed for the good run of the contest, 
				\item configuring for maximum security the switches and routers used so that the contest is held in a fair way,

			\end{itemize}
			\item position: technical support 
				\begin{itemize}
				 \item event: CISCO Expo 2006, Bucharest, România
                                 \item as a member of the technical crue from the Cisco Academy from the Faculty of Automatics and Computer Science, PUB.
                                \end{itemize}
			\item position: member of the netgroup at the Villum Kann Rasmussen Kollegiet (danish dormitory)
				\begin{itemize}
					\item period: March 2008 - October 2009 
					\item attributions: responsable for managing, monitoring, troubleshooting the network equipment, the local server, installing new equipment, providing Internet connection to the inhabitants of the kollegium, assuring that the access policies are respected.
				\end{itemize}
			\item part-time student job \textbf{Motorola Denmark} in the network transport team
				\begin{itemize}
					\item period: May 2008 - October 2009
					\item attributions: 
					\begin{itemize}
						\item managing the development lab of the network transport team, 
						\item performance testing for TETRA transport network, 
						\item development of a packet generating testing system, 
						\item development of technical guidelines for new features
					\end{itemize}
				\end{itemize}
		\end{itemize}	
	\subsection{VoIP engineer}
		\begin{itemize}
			\item position: member of the SQA team
			\item company: 4psa
			\item expertise area: voip 	
			\item period: March 2010 - present
			\item responsabilities: 
				\begin{itemize}
					\item testing of the various components of the voip application (unified communications server) 
					\item testing scenario 
					\item verifing, finding, checking bugs, or find solutions for fixing them
					\item developing and/or improving testing tools
				\end{itemize}

		\end{itemize}

	\subsection{Independent Projects}
		\subsubsection{Web programming}
		\begin{itemize}
			\item \textbf{parohiaortodoxacopenhaga.dk} - Web design and programming for the web site of the Romanian Orthodox Church in Copenhagen 
		\end{itemize}

		\subsubsection{Graphical design}
		\begin{itemize}
			\item Flyers for \textbf{Seishinkan Dojo} - Graphical design of the flyers for the martial arts club Seishinkan in Copenhagen
		\end{itemize}
		
		
		
\section{Events and conferences attended}
	\begin{itemize}
		\item \textbf{CISCO Expo:}  24-25 March 2004 – guest,
		\item \textbf{IDC IT Security Roadshow 2006:} 09.02.2006, representing CISCO Romania,
		\item \textbf{CISCO Expo:} 10-11 April 2006 – Bucharest – Romania – guest and tech support team of the event,
		\item \textbf{IDC Storage Roadshow CEE 2006}: 09.05.2006 – representing CISCO Romania. 
		\item \textbf{IDC IT Security Roadshow 2007} 
		\item \textbf{CISCO Expo 2007} – on an invitation by Cisco Romania
		\item \textbf{Cisco RoNetAcad Conference 2007} – representing CATC Romania
		\item \textbf{Mobile Linux Hacker Workshop} - Nokia Denmark - October 2007
		\item \textbf{Summerschool in Symbolic dynamics and homeomorphisms of the Cantor set} - University of Copenhagen, June 23-27, 2008  
	\end{itemize}

\newpage
\section{Aptitudes}
	\begin{itemize}
	 	\item \textbf{Programming languages: } C (medium), C++ (medium), Java (medium), Assembler (beginner), C\# (beginner), Python (medium-advanced), Prolog (beginner), Mathlab (beginner), Bash (beginner), CafeOBJ (beginner), Erlang (medium-advanced)
		\item \textbf{Web design:} HTML (medium-advanced), Javascript (medium), XML (medium-advanced), CSS (medium),
		\item \textbf{Databases:} Oracle (basic), PostgreSQL (basic),
		\item \textbf{Operating systems (operating, administrating and programming): }
			\begin{itemize}
			 	\item FreeBSD (medium), GNU/Linux (Redhat/CentOS, Debian/Ubuntu) (medium-advanced), Microsoft Windows (basic), Symbian S60 (medium)
			\end{itemize}

		\item \textbf{Revision Control Software:}
			\begin{itemize}
				\item Perforce
				\item Mercurial
			\end{itemize}

		\item \textbf{Networking:}
			\begin{itemize}
			 	\item advanced knowledge of designing and configurating large scale networks, 
				\item network security advanced knowledge, 
				\item very good knowledge of the protocols: TCP/IP, IPv6, RIP, OSPF, IS-IS, STP, RSTP, IEEE 802.1q, VTP, PPP, Frame-Relay, ISDN, RADIUS, DIAMETER, EAP, IEEE 802.1x , IPsec, GMPLS, ASON, SIP, XMPP
			\end{itemize}

		\item \textbf{VoIP:}
			Expertise with:
			\begin{itemize}
				\item UAS (servers): Kamailio, Asterisk
				\item UAC (softphones): x-lite, sjphone, etc
				\item UAC (phones): Snom, Aastra, Grandstream, Polycom, Cisco
				\item testing: sipp 
			\end{itemize}

			Advanced knowledge about the protocols in the SIP and XMPP families:
			\begin{itemize}
				\item voice
				\item presence
				\item extra services: music on hold, voicemail, ivr, call routing  
			\end{itemize}			

	\end{itemize}
\section{Hobbies and scientific interests}
	\begin{itemize}
		\item natural and programming languages,
		\item networking,
		%\item cryptography,
		\item algebra and superior logics,
		\item history, 
	%	\item go%, aikido.
	\end{itemize}
\end{document}          
